% vim: set sw=4 ts=4 et

\chapter{Overview}

\projectname{ }implements a configuration-compatible AY-3-8930 \autocite{AY38930ds} and YM2149 \autocite{YM2149ds} synthesizer core in Amaranth HDL.  It starts in the the AY-3-8910 compatibility mode and the mode select to AY-3-8930 expanded capability mode, as per the AY-3-8930.

\projectname{ }uses information from the Mame source code, but does not use the Mame source code itself and is not supported by Mame.

As per the AY-3-8930, writing $'b101$ to R15 B7-B5 selects the 8930 expanded mode.  In expanded mode, writing $'b1010$ to R15 B7-B4 selects register bank 0, and writing $'b1011$ to R15 B7-B4 selects register bank 1.  Additionally, writing $'b0001$ to R15 B7-B4 selects YM2149F mode.

\begin{tabular}{|c|c|c|}
    \hline
    Mode & B7-B5 & B4 \\
    \hline
    8910 & All undefined values & x \\
    \hline
    8910 & 'b000 & 'b0 \\
    \hline
    YM2149F & 'b000 & 'b1 \\
    \hline
    8930 extended & 'b101 & x \\
    \hline
    Enhanced & 'b110 & x \\
    \hline
    Enhanced Page 3-4 & 'b111 & x \\
    \hline
\end{tabular}

\section{Enhanced Mode}

Enhanced mode extends the duty cycle registers to 8 bit and adds amplitude modulation, vibrato, and duty cycle modulation via three LFOs indexed 0 (disable), 1, 2, or 3.  A chorus effect is also provided, which when enabled generates a duplicate signal for the given channel, without vibrato, such that applying vibrato to the channel creates a chorus effect.  Three Chamberlin state variable filters (SVF) are also provided.  Each channel can select directly from the ramp, square, or triangle waveforms.

The envelope counter is 8 bit in enhanced mode.  Levels can be masked by a priority decoder, converting a 3-bit input into an AND mask.  The 3-bit input is decoded by a binary decoder; each output is ORed with the next-lower output.  For example, an input of $'b101$ (5) sets bit 5, which causes bits 6 and 7 to be set, resulting in $'b1110000$.  This result is ANDed with the envelope counter to produce the final envelope count, in this case producing a 3-bit counter by masking off the lower 5 bits.

These extensions greatly expand the range of timbres produced by the AY-3.  AM and fine-grained duty cycle control can create a large amount of harmonic distortion effects; the vibrato effect is the foundation for the chorus effect, which creates a glassy tone.  Masking the envelope counter can achieve the behavior of the AY-3-8910's 4-bit EC (mask=$'b100$) or the AY-3-8930's 5-bit extended mode counter (mask=$'b011$), or other timbres.

The Chamberlin state variable filters digitally model the nonlinearity of many older sound generators such a the 2A03, SN76489, AY-3-8930, and SID, and expose the internal registers to the user.  There are no modulation controls over the registers; all poles, zeroes, and resonances must be directly set, and values must be calculated by the programmer.

These enhanced features require minimal additional circuitry, and would be appropriate for a 1980-era digital sound generator.

Beyond this, the address bits from DA7-DA4 act as a register file selector; this is functionally equivalent to their original use as a chip select, but only one set of internal circuitry is used to produce the tones for all files.  The Enhanced mode can support up to 16 register files; however, mode select can only be done on register file zero.  Writing to non-existent register files and then reading the value back will return zeroes.  This allows for up to 48 independent channels to be implemented, or as few as only 3 independent channels by not implementing any register files beyond file zero, with a method for software to query the chip's capabilities.

\section{Block Diagram}

\begin{figure}[h!t]
    \centering
    % vim: sw=4 ts=4 et

\begin{tikzpicture}[
        every node/.append style={align=center, minimum width=2cm, minimum height=2cm,fill=white},
        straight arrow/.style={-{Latex[length=2mm]}},
        symbol node/.append style={minimum width = 0.0cm, minimum height=0.0cm, inner sep=-0.1cm, outer sep=0cm, align=center, draw=none},
        DPW/.append style={fill=Emerald!50},
        differencer/.append style={minimum height=1cm,fill=Red!45}
    ]
    \footnotesize
    \matrix[draw, column sep=0.5cm, row sep=0.5cm, nodes=draw, fill=SkyBlue] (Tone Generator)
    {
    \node (registers) {Control\\Registers};
      & \node[minimum height=1cm, fill=Peach!50] (accumulator) {Phase\\Accumulators};
      & \node[draw=none, outer sep=0, minimum width=0, fill=none] (Tone LFO) {from\\LFO};
      & & \\
     & \node[DPW] (DPW4) {DPW4\\Polynomial};
       & \node[symbol node, circle, font={\Huge}] (DPW4 Sum) {$\oplus$};
       & \node[differencer] (DPW4 Differentiator) {Differentiator};
       & \node[symbol node, circle, font={\Huge}] (DPW4 Scale) {$\otimes$}; \\
    };
    \draw[straight arrow] (accumulator.south) -- (DPW4.north);
    \draw[straight arrow] (registers.east) -- (accumulator.west);
    \draw[straight arrow] (Tone LFO.west) -- (accumulator.east);

    \draw[straight arrow] (DPW4.east) -- (DPW4 Sum.west);
    \draw[straight arrow] (DPW4.east) ++(0,-0.75cm) -| (DPW4 Sum.south);
    \draw[straight arrow] (DPW4 Sum.east) -- (DPW4 Differentiator.west);
    \draw[straight arrow] (DPW4 Differentiator.east) -- (DPW4 Scale.west);

    %%% LFO and noise generator

    \matrix[column sep=0.5cm, row sep=0.5cm, nodes=draw, anchor=north west]
    (LFO Section) at ($(Tone Generator.south west) + (0, -1cm)$)
    {
        \node(Noise Shifter) {Noise Shifter}; & \node[fill=Black,text=white] (LFO) {\textbf{LFO}}; \\
    };

    \matrix[draw, column sep=0.5cm, row sep=0.5cm, nodes=draw, anchor=north east, fill=Green]
    (Envelope Generator Section) at ($(Tone Generator.south east) + (0, -1cm)$)
    {
        \node (Envelope Counter) {Envelope\\Counter}; & \node(Envelope Mask) {Mask\\Generator}; \\
    };

    \draw[straight arrow] (Envelope Counter.east) -- (Envelope Mask.west);

    \node[symbol node, circle, font={\Huge}, anchor=center] (Noise Sum)
      at ($(Tone Generator.south east) + (1cm,-0.5cm)$){$\oplus$};
    \draw[straight arrow] (Noise Shifter.north) |- (Noise Sum.west);
    \draw[straight arrow] (DPW4 Scale.east) -| (Noise Sum.north);

    \node[symbol node, circle, font={\Huge}, anchor=center] (Envelope Multiplier)
    at (Noise Sum|-Envelope Mask) {$\otimes$};
    \draw[straight arrow] (Noise Sum.south) -- (Envelope Multiplier.north);
    \draw[straight arrow] (Envelope Mask.east) -- (Envelope Multiplier.west);

    \node[draw, anchor=north west, fill=Apricot] (Filter)
      at ($(LFO Section.south west) + (0, -1cm)$) {State\\Variable\\Filter};
    \node[symbol node, circle, font={\Huge}, anchor=center] (AM Multiplier)
      at (LFO|-Filter) {$\otimes$};

    \draw[straight arrow] (LFO.south) -- (AM Multiplier.north);
    \draw[straight arrow] (Envelope Multiplier.south) |- (AM Multiplier.east);
    \draw[straight arrow] (AM Multiplier.west) -- (Filter.east);
    \draw[straight arrow] ($(Filter.east) + (0,-0.5cm)$) -- ++(10cm,0);

\end{tikzpicture}

    \caption{\label{fig:block-diagram} Full block diagram.  The interpolator generates four waveforms per channel, which are summed and then differentiated.  Triangle generation bypasses the interpolator internally in favor of the naive triangle.}
\end{figure}

\section{Tone Generation}

Each channel can generate a ramp, pulse, or triangle waveform.  To compensate for the aliasing in digital waveform generation, various approaches were considered, including the differential polynomial waveform (DPW) method\autocite{Valimaki2010}.  Ultimately, these methods did not work well with non-static fundamental frequencies, such as with vibrato or pitch bends.  Therefor, a wavetable of band limited waveforms is used.

The wavetable only contains ramps.  Square and PWM are generated by subtracting a ramp from itself phase shifted, with the shift representing the duty cycle.  One wavetable exists beginning at 16Hz and up to 2,048Hz, plus at 3,072Hz, each containing a 1024-entry half-ramp.  The ramp is symmetrical and mirrored over the x axis, similar to a quarter sine, so one ramp contains 2,048 samples.

The interpolator uses linear interpolation with one subtraction and one multiply as a low-pass filter to remove high frequency noise introduced by the discontinuities between samples.  It further from the two involved tables when changing the frequency of a waveform across a power-of-two boundary, and uses the mean average of the two values.

The wavetable entries are sums of sines of each harmonic divided by its order, with the highest sine being no more than 20,000Hz.  At 48,000Hz sampling rate, a 20,000Hz sine will alias at 16,000Hz; at 96,000Hz sampling rate, aliasing doesn't fall below 24,000Hz until a frequency component of 72,000, and no aliasing occurs until 48,000Hz.  By filtering the final output at a cutoff frequency at or below 20,000Hz, aliasing can be eliminated even when decimated to 48kHz $f_s$.

For the triangle waveform, the interpolator bypasses the wavetable and uses the naive triangle instead, as the sinusoidal components have amplitude of the inverse square of their harmonic order.  The 5thth harmonic of C8, 20,945Hz, remains below the Nyquist limit; while the 7th harmonic has an amplitude of $\frac{1}{49}$.  This is 3dB for 24-bit sample size, easily filtered even at 24,000Hz.

The interpolator generates two waveforms per each of three channels in the case of PWM, summing two ramps; and generates two such outputs per channel in the case of the chorus effect.  This requires a total four waveform states per each of three channels, twelve waveforms generated in all.

% Best estimate right now is 7 clock cycles to generate 1 channel tone

% Total should be:
%  7 to generate 1 channel tone
%  1 to add the noise (generated during tone gen)
%  3 to apply envelope and AM (calculated during tone gen, if done within
%    8 cycles)
%  15 (high guess) for SVF
%  2 for bit crusher with QE remediation
%  ??? final output filter
% A total 28 cycles before sending through the final output filter; for 16 parts with
% 48 total channels, this gives 76 cycles, a minimum 6MHz.  With 8 I/Os to configure
% one full channel, that requires 84 total cycles per channel or 8MHz; a full
% reconfigure would demand a 43MHz core.  A 9.216MHz (96kHz*96) crystal gives well
% enough time to complete all tasks.  24.576MHz (96kHz*256) is another possibility for
% a more complex chip incorporating this PSG as a feature.

\section{Noise Shifter}

The noise shifter works in the same way as the AY-3-8930, with AY-3-8910 mode configuration in compatibility mode.

% TODO:  Fully explain and diagram the noise shifter.
\section{Envelope Generator}

The envelope generator is identical to the AY-3-8930 EG, except that in enhanced mode it has an 8-bit envelope counter that can be masked to fewer bits.

\section{State Variable Filter}

% TODO:  SVF with Chamberlin or LT architecture

\section{Bitcrusher}

Output will have a LFO-controlled bitcrusher masking the lower bits of each sample to zero.  The bitcrusher will include a quantization error compensation option.

\section{Output Filter}

\section{Control Unit}

The control unit connects all of the parts together and coordinates work.

The control unit gets information from the control bus.  This can be an AY-3-8930 pinout interface or another control interface.  The control bus provides the clock, configuration I/O, and sound output.

The control unit also manages modulation by the LFO.  It iterates through each channel and copies the value register for the LFO referenced by the tone generator into the tone generator's vibrato register, then copies the LFO referenced by the envelope generator into the envelope generator's amplitude modulation register.

The control unit sums the outputs of the tone generator and noise generator, then fetches the gain value from the envelope generator and multiplies it by this sum.

The control unit sends the gain-adjusted channel data to the state variable filter.

The control unit runs the filtered channel data through the pre-filter bitcrusher, then through the output filter's first pass, through the post-filter bitcrusher, then through the output filter's second pass.  This requires two states for the output filter; and either filter may be bypassed.

Several things occur in parallel:

\begin{itemize}
	\item Tone generation, noise, final gain calculation, and LFO computation all occur at the same time
	\item LFO reconfiguration is allowed immediately after the last LFO calculation
	\item Construction of the gain-adjusted signal is pipelined
	\item Envelope reconfiguration is allowed immediately after the last channel's gain is applied
	\item State variable filter, bitcrusher, and output filter are all pipelined, and are a continuation of the pipeline constructing the gain-adjusted signal 
\end{itemize}

This allows the production of 96kHz output with a clock as low as 3MHz for 3 channels, without configuration time; and as low as 8MHz for 48 channels with time to configure one full channel each sample (0.5ms to configure all 48, although precise programming timing can reduce this to under 0.1ms).

The recommended clock rates are 9.216MHz (96kHz*96) or 24.576MHz (96kHz*256).  A high-speed serial programming interface allows alternate configuration, fully configuring the tone generator, envelope generator, and SVF for a selected channel in one clock cycle, giving complete control over each individual's sample generation even at the sub-10MHz core clock rate.