% vim: set sw=4 ts=4 et

\chapter{Overview}

\projectname{ }implements a configuration-compatible AY-3-8930 and YM2149F synthesizer core in Amaranth HDL.  It starts in the the AY-3-8910 compatibility mode and the mode select to AY-3-8930 expanded capability mode, as per the AY-3-8930.

\projectname{ }uses information from the Mame source code, but does not use the Mame source code itself and is not supported by Mame.

As per the AY-3-8930, writing $'b101$ to R15 B7-B5 selects the 8930 expanded mode.  In expanded mode, writing $'b1010$ to R15 B7-B4 selects register bank 0, and writing $'b1011$ to R15 B7-B4 selects register bank 1.  Additionally, writing $'b0001$ to R15 B7-B4 selects YM2149F mode.

\begin{tabular}{|c|c|c|}
	\hline
	Mode & B7-B5 & B4 \\
	\hline
	8910 & All undefined values & x \\
	\hline
	8910 & 'b000 & 'b0 \\
	\hline
	YM2149F & 'b000 & 'b1 \\
	\hline
	8930 extended & 'b101 & x \\
	\hline
	Enhanced & 'b110 & x \\
	\hline
	Enhanced Page 3-4 & 'b111 & x \\
	\hline
\end{tabular}

\section{Enhanced Mode}

Enhanced mode extends the duty cycle registers to 8 bit and adds amplitude modulation, vibrato, and duty cycle modulation via three LFOs indexed 0 (disable), 1, 2, or 3.  A chorus effect is also provided, which when enabled generates a duplicate signal for the given channel, without vibrato, such that applying vibrato to the channel creates a chorus effect.  Three Chamberlin state variable filters (SVF) are also provided.

The envelope counter is 8 bit in enhanced mode.  Levels can be masked by a priority decoder, converting a 3-bit input into an AND mask.  The 3-bit input is decoded by a binary decoder; each output is ORed with the next-lower output.  For example, an input of $'b101$ (5) sets bit 5, which causes bits 6 and 7 to be set, resulting in $'b1110000$.  This result is ANDed with the envelope counter to produce the final envelope count, in this case producing a 3-bit counter by masking off the lower 5 bits.

% TODO:  Do we need a ramp waveform, considering you can emulate saw and triangle with the EG?

These extensions greatly expand the range of timbres produced by the AY-3.  AM and fine-grained duty cycle control can create a large amount of harmonic distortion effects; the vibrato effect is the foundation for the chorus effect, which creates a glassy tone.  Masking the envelope counter can achieve the behavior of the AY-3-8910's 4-bit EC (mask=$'b100$) or the AY-3-8930's 5-bit extended mode counter (mask=$'b011$), or other timbres.

The Chamberlin state variable filters digitally model the nonlinearity of many older sound generators such a the 2A03, SN76489, AY-3-8930, and SID, and expose the internal registers to the user.  There are no modulation controls over the registers; all poles, zeroes, and resonances must be directly set, and values must be calculated by the programmer.

These enhanced features require minimal additional circuitry, and would be appropriate for a 1980-era digital sound generator.

Beyond this, a 4-bit read-write register in the last register page acts as a register file selector.  When loaded with a value, it checks if that value is greater than the total register files; if so, it loads the register with the highest register file available.  The Enhanced mode can support up to 16 register files, which act as independent AY-3-8930 Enhanced tone generators, making the register file select analogous to a chip select; however, mode select can only be done on register file zero.  This allows for up to 48 independent channels to be implemented, or as few as only 3 independent channels by not implementing any register files beyond file zero, with a method for software to query the chip's capabilities.

\section{Block Diagram}

% TODO:  Block diagram goes here.
%
% Major components:
%
%  - Control registes
%  - Envelope generator
%    - Envelope counter
%    - Mask generator
%  - Noise shifter
%  - Tone generators
%    - Phase accumulators
%    - Differentiated Polynomial Generator
%  - LFOs
%  - Filters
%
% Components of the Differentiated polynomial generator:
%
%  - Phase input
%  - LFO input
%  - Sample register (holds last sample and differences)
%  - Differentiator
%  - Scaling correction
%
% Differentiated polynomial generator handles a lot of the tone shaping:
%
%  - The pulse wave duty cycle is controlled by generating two saw waves,
%    and phase-shifting one of these.
%  - Duty cylcle modulation is handled by phase modulation of one of the
%    saw waves.
%  - Vibrato is handled by frequency modulation, slightly reducing or
%    increasing the rate of phase accumulation.
%  - Chorus is handled by wet/dry mixing of vibrato and non-vibrato
%    waveforms, meaning four saw waves are generated per channel.
%  - This also means saw waves can be generated by not mixing the pulse
%    wave component.
%  - A square wave (50% duty cycle pulse) can be fed to a simple
%    integrator and an accumulator, then given frequency-dependant
%    scaling, to produce a triangle wave.
%
% This is relatively complex because digitally-generated trivial
% waveforms have severe aliasing; see Välimäki et al for DPW alias
% suppression.  Fortunately, this is just a matter of generating a
% large quantity of waveforms in a pipeline, not extra hardware beyond
% basic square waveform generation.  More phase accumulators and sample
% registers are needed for each channel, but not more computation
% circuitry.
%